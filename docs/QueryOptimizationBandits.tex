\documentclass[12pt]{article}
\usepackage[utf8]{inputenc}
\usepackage[english,russian]{babel}
\usepackage[T1]{fontenc}
\usepackage{amsmath,amsfonts,amssymb}
\usepackage{graphicx}
\usepackage{a4wide}
\title{Бандиты в Query Selection}
\author{Шестаков Владимир}
\date{}
\begin{document}
\maketitle

\begin{abstract}
В работе исследуются способы решения задачи Query Optimization. Рассматриваются конкретные решения, использующие метод многоруких бандитов. Среди таких решений выделяется одно конкретное с названием Bao \cite{bao}, использующее семплирование Томпсона. Данное решение требует проверки на практике, а также исследования на возможность доработки.
\end{abstract}
\paragraph{Keywords:} Query Optimization, Multiarmed Bandits, Thompson sampling

\section{Введение}
В современном мире требуются новые инструменты для эффективной обработки данных. В основе подобных систем лежат базы данных, но с увеличением сложности возникает задача оптимизации запросов. Методы, использующие статические правила и эвристики, оказываются не способными выдержать динамические изменения. Современные подходы, использующие машинное обучение, сталкиваются с рядом ограничений: длительное время обучения; неспособность адаптации к изменениям в данных; проблема "хвостовых" случаев, связанная с тем, что запросы исполняются быстро в среднем, но даже редкие случаи длительного исполнения неприемлемы в реальном мире. Таким образом, многие методы становятся непрактичными для их использования.

Среди различных методов выделяется Bao (Bandit Optimizer) \cite{bao} — один из способов оптимизации запросов, который успешно преодолевает указанные ограничения. В отличие от других подходов, Bao не заменяет традиционный оптимизатор, а дополняет его, работая как надстройка над уже существующим. Основная идея, заложенная в Bao, это выделение нескольких различных подсказок — изменений в текущем запросе, — которые оптимизатор может применить. Некоторые конечные подмножества таких подсказок используются как "руки" в методе многорукого бандита. Таким образом, сформулировав задачу как контекстного многорукого бандита, Bao использует семплирование Томпсона для нахождения лучшего набора подсказок среди предложенных.

В статье будет воспроизведена работа Bao, проверена его эффективность на тестовых данных и предложены возможные улучшения. Цель работы — показать, насколько различные стратегии решения задачи контекстного многорукого бандита эффективны для оптимизации запросов.

\section{Постановка задачи}

Для начала введём несколько определений, относящихся к самой задаче Query Selection — выбор оптимальной стратегии для определённого запроса.

Каждый набор подсказок является функцией, сопоставляющей запросу $q \in Q$, где $Q$ — множество всевозможных запросов, дерево плана запроса $t \in T$, где $T$ — все деревья плана запроса, какие только могут быть. Обозначим соответствующий набор подсказок как $HSet_i$. Все такие наборы подсказок должны содержаться в одном определённом семействе $F$, фиксированном изначально. Обозначим метрику, по которой будет происходить оптимизация, как функцию $P : T \xrightarrow{} R$, т.е. действующую из множества деревьев плана запроса в множество действительных чисел. Также обозначим функцию выбора набора подсказок как $B: Q \xrightarrow{} F$. Нам нужно минимизировать значение функции, так что определим потерю $R_q$ как $(P(B(q)(q)) - \min\limits_{i} P(HSet_i))^2$. Соответственно, получаем уже выпуклую функцию, которую проще оптимизировать. Все обозначения, что были в исходной статье, здесь повторяются с таким же смыслом.

Теперь определим то, как выбирается стратегия. Пусть дано множество моделей $M$, предсказывающих по запросу, какую из рук дёрнуть, т.е. на запрос $q$ выдают определённый набор подсказок (условно будем считать его числом от 1 до $|F|$ и сопоставлять числу $i$ набор подсказок $HSet_i$). Проверяя выбранный план запроса, значения параметров модели обновляются. Среди таких моделей рассмотрим несколько фиксированных, соответствующих конкретным стратегиям многорукого контекстного бандита. Заодно получаем ответ, каким образом получается задача о контекстном бандите: контекстом является сам запрос, множество стратегий (наборы подсказок) служат руками такого бандита.

Одним из кандидатов в такие модели, предложенный в статье, является стратегия, основанная на семплировании Томпсона. Для этого внутри стратегии используется модель, сопоставляющая деревьям плана запроса значения метрики, оптимизируемой в задаче, а затем среди таких значений выбирается минимальное. Как конкретно устроена данная стратегия можно прочитать в исходной статье\cite{bao}. Разберём кроме этого несколько иных стратегий.

TODO: Добавить другие стратегии и их описание.

\bibliographystyle{unsrt}
\bibliography{QueryOptimizationBandits}
\end{document}